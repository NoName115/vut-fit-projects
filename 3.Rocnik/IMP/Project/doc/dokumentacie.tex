\documentclass{article}

\usepackage[utf8]{inputenc}
\usepackage[slovak]{babel}
\usepackage{geometry}
\usepackage{url}

\geometry{a4paper, left=25mm, right=25mm, top=20mm}
\bibliographystyle{czplain}

\begin{document}

\title{ESP8266: snímání teploty (IoT, cloud) - IMP 2016/2017}
\author{Róbert Kolcún, xkolcu00}
\maketitle

\section{Cieľ projektu}
Cieľom projektu je vytvoriť jednoduchý systém pre snímanie a následné zobrazovanie teploty.
Pre projekt je použité zariadenia ESP8266 a teplotním čidlom DS18B20.
Ako klient ma vystupovať zariadenie ktoré sníma teplotu a tieto hodnoty zasiela na server.
Server tieto hodnoty zberá, aj od niekoľkých zariadení naraz, následne umožnuje ich zobrazenie za pomoci jednoduchej webovej stránky.

\section{Popis ovládania}
\subsection{ESP8266}
Pre plnú funkcionalitu je potrebné na zariadenie nahrať micropython \cite{odkaz:micropython} a následne súbory
z odovzdaného archívu \textit{boot.py} a \textit{esp\_class.py}

Script \textit{boot.py} zabezpečuje spustenie merania teploty pri zapnutí zariadenia
alebo po jeho resetovaní.

Následne súbor \textit{esp\_class.py}, kde je implementovaná celá funkcionalita zariadenia.
Pre správne fungovanie je následne už len potrebné editovať triedu v súbore \textit{esp\_class.py} kde je potrebné
nastaviť meno, heslo Wi-Fi, IPv4 adresu a číslo portu na ktoré sa ma zariadenia pripojiť a zasielať dáta o teplote.

\subsection{Server}
Funkcionálna časť servera je implementovaná v súboroch \textit{http\_server.py} a \textit{server.py},
pre spustenie webového servera a socket server stačí sputiť script \textit{main.py}, ktorý spustí paralelne 2 procesy, každý pre daný server.

Pre zmenu portu na socket servery je potrebné upraviť premennu PORT v súbore \textit{server.py} a pre IP adresu a port na ktorom
má bežat http server je potrebne upraviť premenné PORT\_NUMBER a HOST\_IP v súbore \textit{http\_server.py}.

\section{Bezpečnosť}
Zariadenia zasiela na server 2 informácie svoje ID a teplotu, táto komunikácia nieje nijak šifrovaná ani zabezpečená.
Pre účeli demoštrácie si zariadenia vygeneruje svoje ID náhodne.

V reálnom použití by bolo potrebné každé zariadenia pripojené do siete identifikovať unikátnym ID.
Pre bezbečnú komunikácia medzi serverom a klientom by bolo možné použiť TLS/SSL protokol, avšak nevýhoda tohto protokolu
je že funguje iba nad TCP, preto existuje verzia tohto protokolu pracujúca nad UDP a to DTLS\cite{odkaz:iotsecurity}.

Taktiež by bolo vhodné použiť cloude systémi ako npr. AWS od Amazonu ktoré majú priamu podporu
pre IoT zariadenia, kde podporujú aj komunikačné protokoly ako MQTT ktoré boli navrhnuté špecificky pre IoT svet.
Tieto systému disponujú a riešia bezpečnosť komunikacie pomocou podpisovania správ.

\renewcommand\refname{Odkazy}
\bibliography{literatura}

\end{document}
